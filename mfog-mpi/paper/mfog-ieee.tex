\documentclass[conference]{lib/IEEEtran}
\IEEEoverridecommandlockouts
% The preceding line is only needed to identify funding in the first footnote. If that is unneeded, please comment it out.
\usepackage{cite}
\usepackage{amsmath,amssymb,amsfonts}
\usepackage{algorithmic}
\usepackage{graphicx}
\usepackage{textcomp}
\usepackage{xcolor}
% \def\BibTeX{{\rm B\kern-.05em{\sc i\kern-.025em b}\kern-.08em
%     T\kern-.1667em\lower.7ex\hbox{E}\kern-.125emX}}
\begin{document}

\title{Edge Intrusion Detection with Distributed Novelty Detection: Design, Implementation and Evaluation
\thanks{CNPq}
}

\author{
  \IEEEauthorblockN{Luís Puhl, Hermes Senger, Guilherme Weigert Cassales}
\IEEEauthorblockA{Universidade Federal de São Carlos, Brasil \\
  Email: \{luispuhl, gwcassales\}@gmail.com, hermes@ufscar.br
}
% \IEEEauthorblockN{Luís Puhl}\IEEEauthorblockA{\textit{Universidade Federal de São Carlos}, Brasil \\luispuhl@gmail.com}\and
% \IEEEauthorblockN{Hermes Senger}\IEEEauthorblockA{\textit{Universidade Federal de São Carlos}, Brasil \\hermes@ufscar.br}\and
% \IEEEauthorblockN{Guilherme Weigert Cassales}\IEEEauthorblockA{\textit{Universidade Federal de São Carlos}, Brasil \\gwcassales@gmail.com}
% \and
% \IEEEauthorblockN{4\textsuperscript{th} Given Name Surname}
% \IEEEauthorblockA{\textit{dept. name of organization (of Aff.)} \\
% \textit{name of organization (of Aff.)}\\
% City, Country \\
% email address or ORCID}
}

\maketitle

\begin{abstract}
  
  The implementation of the Internet of Things (IoT) is sharply increasing the small
  devices count and variety on edge networks and, following this increase the
  attack opportunities for hostile agents also increases, putting more pressure
  on the network administrator's need for tools to detect and react to those
  threats.

  One such tool are the Intrusion Detection Systems (IDS) where the network
  traffic is captured and analysed raising alarms when a known attack pattern or
  new pattern is detected. To build an IDS one option for base algorithm are
  the Data Stream (DS) Novelty Detection (ND) being MINAS one of those.
  
  Furthermore, for a network security tool to operate in the context of edge and
  IoT it has to comply with processing time, storage space and energy
  requirements alongside with traditional requirements for stream and network
  analysis like accuracy and scalability.

  This paper addresses the construction details and evaluation of an prototype
  distributed IDS using MINAS ND algorithm.
  We discuss the algorithm steps, how it can be deployed in a distributed architecture,
  the impacts on the accuracy of MINAS and evaluate the performance and scalability
  using a cluster of constrained devices commonly found in IoT scenarios.

  We found an increase of \textit{0.0 y} processed network flow descriptors per core
  added to the cluster. Also \textit{0.0 x1\%} and \textit{0.0 x2\%} change in
  \textit{F1Score} in the tested datasets when stream was unlimited in speed and
  limited to \textit{0.0 z MB/s} respectively.

% This document is a model and instructions for \LaTeX.
% This and the IEEEtran.cls file define the components of your paper [title, text, heads, etc.].
% *CRITICAL: Do Not Use Symbols, Special Characters, Footnotes,
% or Math in Paper Title or Abstract.
\end{abstract}

\begin{IEEEkeywords}
% Detecção de Novidades, Detecção de Intrusão, Fluxos de Dados, Computação Distribuı́da, Computação em Névoa, Internet das Coisas.
novelty detection, intrusion detection, data streams, distributed system, edge computing, internet of things
\end{IEEEkeywords}

\section{Introduction}
% This document is a model and instructions for \LaTeX.
% Please observe the conference page limits. 

\section*{Acknowledgment}

% The preferred spelling of the word ``acknowledgment'' in America is without an ``e'' after the ``g''.
% Avoid the stilted expression ``one of us (R. B. G.) thanks $\ldots$''.
% Instead, try ``R. B. G. thanks$\ldots$''.
% Put sponsor  acknowledgments in the unnumbered footnote on the first page.
The authors thank CAPES (Coordenação de Aperfeiçoamento de Pessoal de Nível Superior - Código de Financiamento 001).
Hermes Senger also thanks CNPQ (Contract 305032/2015-1) and FAPESP (Contract 2018/00452-2, and Contract 2015/24461-2) for their support.

\bibliography{refs.bib}
\end{document}
